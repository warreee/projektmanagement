\documentclass[11pt,a4paper]{article}
\usepackage[utf8]{inputenc}
\usepackage[german]{babel}
\usepackage{amsmath}
\usepackage{amsfonts}
\usepackage{amssymb}
\usepackage{graphicx}
\usepackage[left=2cm,right=2cm,top=2cm,bottom=2cm]{geometry}
\author{Ward Schodts}

% Voor frames
\usepackage{mdframed}
% Voor kleuren
\usepackage{xcolor}

\definecolor{solarback}{HTML}{d9d9d9}
\definecolor{solarfront}{HTML}{000000}
\mdfdefinestyle{de}{linewidth=0pt,backgroundcolor=solarback,fontcolor=solarfront}
\newenvironment{de}[1]
{\begin{mdframed}[style=de]\begin{mydef}{\textbf{#1:}}\\} 
{\end{mydef}\end{mdframed}}
\newtheorem{mydef}{Definition}
\usepackage{float}

\newcommand{\tabitem}{~~\llap{\textbullet}~~}

\usepackage{array}
\newcolumntype{P}[1]{>{\centering\arraybackslash}p{#1}}

% Set graphics path

\graphicspath{{figures/}}
\usepackage{graphicx}
\usepackage{caption}
\usepackage{subcaption}

\begin{document}
\section{Was ist Projektmanagement?}
\subsection{Projekt}

\begin{de}{Projekt (DIN)}
Vorhaben, das im wesentlichen durch  Einmaligkeit der Bedingungen in ihrer Gesamtheit gekennzeichnet ist, wie z.B.:
\begin{itemize}
\setlength\itemsep{0em}
\item Zielvorgabe,
\item zeitlich, finanzielle oder andere Begrenzungen,
\item Abgrenzungen gegenüber anderen Vorhaben,
\item projektspezifische Organisation.
\end{itemize}

\end{de}

\begin{de}{Projekt (allgemein)}
Ein Projekt ist eine besondere, umfangreiche und zeitlich begrenzte
Aufgabe von relativer Neuartigkeit mit hohem Schwierigkeitsgrad und
Risiko, die in der Regel enge fachübergreifende Zusammenarbeit aller
Beteiligten fordert.
\end{de}

\begin{table}[H]
\def\arraystretch{1.2}
\centering
\begin{tabular}{|p{0.2\textwidth}|p{0.7\textwidth}|}

\hline
\textbf{Kriterium} & \textbf{Beschreibung} \\

\hline
Zeitliche Befristung & 
\tabitem Festgelegter Anfangs- und Endzeitpunkt
\\
\hline
Neuartigkeit &
\vspace{-\topsep}
\begin{itemize}
\itemsep0em
\setlength{\itemindent}{-1.3em}
\item Keine sich wiederholenden Routinevorgänge
\item Neue Aufgabestellung mit besonderer Herausforderung
\end{itemize}
\\

\hline
Einmaligkeit &
\vspace{-\topsep}
\begin{itemize}
\itemsep0em
\setlength{\itemindent}{-1.3em}
\item Insgesamt einmaliges Vorhaben
\item Einzelne Aktivitäten innerhalb des Vorhabens können sich jedoch wiederholen
\end{itemize}\\

\hline
Größe & 
\tabitem Umfang müss Einrichtung eines Projektes und einer Projektorganisation rechtfertigen\\
\hline
Komplexität und Zusammenarbeit &
\vspace{-\topsep}
\begin{itemize}
\itemsep0em
\setlength{\itemindent}{-1.3em}
\item Verschiedene untereinander abhänge Teilaufgaben
\item Hoher Abstimmungsbedarf
\item Fächerübergreifende Zusammenarbeit
\end{itemize}\\

\hline
\end{tabular}

\caption{Kriterien für Projekte}
\end{table}

\begin{table}[H]
\centering
\def\arraystretch{1.4}
\begin{tabular}{|P{0.4\textwidth}|P{0.4\textwidth}|}
\hline
\textbf{Routine-Aufgaben} & \textbf{Projekte}\\
wiederholende Tätigkeiten & einmalige Tätigkeiten\\
standardisierbare Tätigkeiten & individuelle Tätigkeiten\\
trainierbare Tätigkeiten & ``do it right the first time'' Anspruch\\
reproduzierbare Ergebnisse & selten reproduzierbare Ergebnisse\\
systematische Fehler leicht erkennbar  & systematische Fehler schwer erkennbar\\
kontinuierbare Verbesserung möglich & sofortige Korrektur notwendig\\
meist abteilungsspezifische Abwicklung & meis interdisziplinäre Abwicklung\\
meist statische Randbedingungen & meist dynamische Randbedingungen\\
\hline
\end{tabular}
\caption{Differenz zwischen Routine und Projekt}
\end{table}

\subsubsection*{Übung}
Schau die Übungsaufgabe (Welche Vorhaben sind Projekt?) noch mal an!
\subsection{Projektmanagement}

\begin{de}{Projektmangement (DIN)}
Gesamtheit von Führungsaufgaben, -organisation, -techniken und
–mittel für die Abwicklung eines Projektes.
\end{de}

\begin{de}{Projektmanagement (allgemein)}
Projektmanagement ist eine Führungskonzeption für direkte
fachübergreifende Koordination von Planung, Entscheidung,
Realisierung, Überwachung und Steuerung bei der Abwicklung
interdisziplinärer Aufgabenstellungen.
\end{de}

\section{Organisation des Projektmanagements}

\subsection{Organisationsstrukturen}

\begin{figure}[H]
	\centering
	\begin{subfigure}{0.45\textwidth}
		\caption*{Hierarchisches Organisationssystem:}
		\includegraphics[width=\textwidth]{hierarchisches}
	\end{subfigure}
	~
	\begin{subfigure}{0.45\textwidth}
		\caption*{Funktionale Organisationsstruktur:}
		\includegraphics[width=\textwidth]{funktionale}
	\end{subfigure}
	\par\bigskip
	\par\bigskip
	\begin{subfigure}{0.45\textwidth}
		\caption*{Spartenorganisation:}
		\includegraphics[width=\textwidth]{sparten}
	\end{subfigure}
	~
	\begin{subfigure}{0.45\textwidth}
		\caption*{Gemischte Organisation (Produkt- und fachbereiche):}
		\includegraphics[width=\textwidth]{gemischte}
	\end{subfigure}
	
	\begin{subfigure}{0.5\textwidth}
		\caption*{Matrix-Organisation:}
		\includegraphics[width=\textwidth]{matrixorganisation}
	\end{subfigure}
\end{figure}

\begin{table}[H]
\centering
\def\arraystretch{1.4}
\begin{tabular}{|P{0.4\textwidth}|P{0.4\textwidth}|}
\hline
\textbf{Zentralisierung} & \textbf{Dezentralisierung}\\
\hline
\hline
Alle Vollmachten liegen bei der Unternehmensleitung & Delegation von Entscheidungen (Entscheidungen vor Ort)\\
\hline
Einheitliche Strategien, Prozeduren und Entscheidungen & Entlastung des Topmanagements\\

\hline
Minimierung der Duplikation von Funktionen & Entwicklung von Generalisten\\
\hline
Reduzierung der Gefahr, dass sich Firmenaktivitäten verselbständigen & Förderung der Zusammenarbeit (Teamgeist)\\
\hline
Umfangreiche Kontroll- und Koordinationsprozeduren nicht erforderlich & Produktspezialisierung\\
\hline
Direkter Einfluss des Topmanagement-Teams auf alle betriebliche Belange & Höhere Effizienz des Managements durch größere Nähe am Schauplatz\\
\hline
\end{tabular}
\caption{Wichtige vorteile der Zentralisierung und Dezentralisierung}
\end{table}

\subsection{Projektmanagement-Arten}
\subsubsection{Reines Projektmanagement}
\begin{minipage}[t]{0.5\textwidth}
	\subsubsection*{Vorteile:}

	\begin{itemize}
	\itemsep0em
		\item Straffe Projektleitung
		\item Klare eindeutige Projektverantwortung
		\item Einheit des Auftrags-Empfangs (Aufgabe \& Kompetenz)
	\end{itemize}
	\subsubsection*{Nachteile:}

	\begin{itemize}
	\itemsep0em
		\item Starre Organisationsform
		\item Hohe Gemeinkosten
		\item Eingliederung zeitlich begrenzter Spezialistentätigkeiten
	\end{itemize}
\end{minipage}
\hspace{0.5cm}
\begin{minipage}[t]{0.45\textwidth}
	\subsubsection*{Struktur:}
	\begin{figure}[H]
	\includegraphics[width=\textwidth]{reines}
	\end{figure}
\end{minipage}
\vspace{1cm}

\subsubsection{Projektmanagement als Stabsfunktion (Einfluss-Projektmanagement)}
\begin{minipage}[t]{0.5\textwidth}
	\subsubsection*{Vorteile:}

	\begin{itemize}
	\itemsep0em
		\item Eingliederung des Projektteams ohne größere organisatorische Eingriffe möglich
		\item Flexibler Personaleinsatz
		\item Relativ gute Nutzung der Personalkapazitäten
	\end{itemize}
	\subsubsection*{Nachteile:}

	\begin{itemize}
	\itemsep0em
		\item Fehlende Weisungsbefugnis
		\item Der Projektleiter hat keine umfassende Projektverantwortung
	\end{itemize}
\end{minipage}
\hspace{0.5cm}
\begin{minipage}[t]{0.45\textwidth}
	\subsubsection*{Struktur:}

	\begin{figure}[H]
	\includegraphics[width=0.95\textwidth]{stabsfunktion}
	\end{figure}
\end{minipage}

\vspace{1cm}
\subsubsection{Matrix-Projektmanagement}
\begin{minipage}[c]{0.5\textwidth}
	\subsubsection*{Vorteile:}

	\begin{itemize}
	\itemsep0em
		\item Flexible Eingliederung der Projektmitarbeiter
		\item Fachverantwortung des Projektleiters
	\end{itemize}
	\subsubsection*{Nachteile:}

	\begin{itemize}
	\itemsep0em
		\item Problematische Kompetenzabgrenzung
		\item Konflikte (zwischen ``Projekt'' und ``Linie'')
	\end{itemize}
\end{minipage}
\begin{minipage}[c]{0.45\textwidth}
	\subsubsection*{Struktur:}

	\begin{figure}[H]
	\includegraphics[width=\textwidth]{matrix1}
	\end{figure}
	
	\begin{figure}[H]
	\includegraphics[width=\textwidth]{matrix2}
	\end{figure}
\end{minipage}

\vspace{1cm}

\subsubsection{Übersicht}

\begin{figure}[H]
	\includegraphics[width=\textwidth]{projektorganisation}
\end{figure}

\subsubsection{Empfehlungen für die Organisationsmodelle}
\begin{table}[H]
\centering
\def\arraystretch{1.2}
\begin{tabular}{|P{0.3\textwidth}|P{0.3\textwidth}|P{0.3\textwidth}|}
\hline 
	\textbf{Projektmanagement als Stabsfunktion} 
	
	& 
	
	\textbf{Matrix-Projektmanagement} 
	
	& 
	
	\textbf{Reines ProjektManagement}
	\\
\hline	
	\vspace{-\topsep}
	\begin{itemize}
	\itemsep0em
	\setlength{\itemindent}{-1.3em}
		\item Kleinere Projekte mit geringem Risiko und zeitlich nicht kritisch
		\item Erarbeitung von unternehmensweiten Kompromissen
		\item Nähe zum Top-Management, welches Bedeutung und Einflussnahme zeigt
	\end{itemize}
	
	&
	
	\vspace{-\topsep}
	\begin{itemize}
	\itemsep0em
	\setlength{\itemindent}{-1.3em}
		\item Hoher Anzahl laufender Projekte mit starkem Abteilungsübergriff (Interdiszplinarität)
		\item Relativ ähnliche Projekte
	\end{itemize}
	
	&
	
	\vspace{-\topsep}
	\begin{itemize}
	\itemsep0em
	\setlength{\itemindent}{-1.3em}
		\item Komplexe, neuartige Projekte
		\item Hohe strategische Bedeutung und starken Zeitdruck
		\item Bei langfristigen Projekten
	\end{itemize}
	\\
\hline	
	

\end{tabular}
\end{table}

\subsection{Kriterien verschiedener Organisationsformen des Projektmanagements}

\begin{figure}[H]
	\includegraphics[width=\textwidth]{kriterien}
\end{figure}

\subsection{Literatur}
Kapitel 2.3.

\pagebreak

\section{Aufgaben des Projektmanagements}

\begin{figure}[H]
\centering
	\includegraphics[width=0.36\textwidth]{hauptaufgaben}
\end{figure}


\subsection{Erfolgreiches Projektmanagements}

\subsubsection{Wesentliche Komponenten des Projektmanagements}
	\begin{figure}[H]
	\begin{enumerate}
	\itemsep0em
		\item \textbf{Klare Zielsetzung:} Sorgfältige Definition von klaren, eindeutigen, realistischen und von den Betroffenen akzeptierten Projektzielen und Zwischenzielen vor Inangriffnahme des Projektes.
		\item \textbf{Topmanagement-Engagement:} Projektunterstützung (einschl. Bereitstellung der erforderlichen Mittel
bzw.Kapazitäten) durch Topmanagement (Unternehmensleitung bzw.Auftrag
geber) und Führungskräfte der beteiligten Unternehmenseinheiten.
	
	\item \textbf{Teamarbeit/Kooperation:} Echte Teamarbeit (Teamgeist) innerhalb des Projektteams einschl. enger
Kooperation mit allen beteiligten Stellen.
\begin{center}
\includegraphics[width=0.5\textwidth]{effiizienteprojektarbeit}
\end{center}


	\item \textbf{Klare Zuständigkeiten:} Festlegung personifizierter Verantwortung und Kompetenz (einschl.
erforderlicher Befugnisse) mit dazugehöriger organisatorischer Regelung
und Überwachung.

	\item \textbf{Effektives Controlling:} Laufende Planung, Überwachung und Steuerung von Leistungsumfang
(Quantität \& Qualität), Zeit, Kosten und Kapazitäten.

	\item \textbf{Prioritäten-Entscheidung:} Laufende Prioritätenfestlegung aller aktuellen Projekte und Aufgaben
vor allem für die Bewältigung von Kapazitätsengpaßstellen.
	\end{enumerate}
	\end{figure}

\subsubsection{Hauptaufgaben}
\textbf{Ziele:} Definition klarere, eindeutiger, erreichbarer und akzeptierter Zieleund Zwischenziele als Basis Aller aktivitäten.
\begin{figure}[H]
\centering
\includegraphics[width=0.95\textwidth]{hauptaufgabendesprojektmanagements}

\end{figure}


\subsubsection{Schwerpunkte für ein erfolgreiches Projektmanagement}

\begin{enumerate}
\item Beherrschung der \textbf{sachlichen} Aspekte für das Projektmanagement
(Organisation, Methoden, Hilfsmittel).

\item Beachtung \textbf{menschlicher Aspekte} beim Projektmanagement,
insbesondere bei der Teamarbeit und bei der Kooperation mit den
Mitwirkenden bzw. Betroffenen.

\item \textbf{Identifizierung} der verantwortlichen Leitung und der Führungskräfte
der betroffenen Einheiten sowie der Mitwirkenden \textbf{mit den Projektzielen}
und den Methoden des Projektmanagements;
d.h. der Erfolg des Projektmanagements hängt von der Kompetenz,
der strategischen Führung und der Unterstützung auf Leitungsebene
ab; denn Projektmanagement ist eine Art von „strategischer
Delegation“, bei der die leitenden Führungskräfte den Projekt-
managern die Kompetenz und die Verantwortung für Projektaufgaben
übertragen.
\end{enumerate}

\subsubsection{Voraussetzungen für den Erfolg eines Projektes}

\begin{enumerate}
\item Starke Einbindung des Auftraggebers bei Projektplanung
und -durchführung.

\item Eine umfassende Machbarkeitsstudie des Projektes während der
strategischen Planungsphase des Unternehmens.

\item Fortlaufende Projektplanung, -koordinierung und -überprüfung.

\item Teamarbeit, die sich aus der Konzentration auf eine klar
umrissene Zielsetzung ergibt.

\item Verpflichtung des Auftraggebers, technische Konstruktions-
entscheidungen, Projektmanagementzielsetzungen und moderne
Managementverfahren zu unterstützen.

\item Sicherstellung, dass die Prinzipien des Projektmanagements
allen Mitarbeitern des Projektteams und den am Projekt
Beteiligten – einschl. Führungskräften – bekannt sind.
\end{enumerate}

\subsection{Der Projektleiter}

\subsubsection{Aufgaben eines Projektleiters}
\begin{itemize}
\item Definition der Aufgabe und Planung eines Projektes, Aufwandschätzung,
Zeitschätzung (Netzplan)
\item Aufgabenverteilung, Abgrenzung der Teilgebiete
\item Festlegung von Methoden und Verfahren
\item Koordination - Projektteam
- Fachabteilungen
- externe Beratung
\item Überwachung von - Projektfortschritt/Leistungsumfang (Quantität/Qualität)
- Zeit (Termine)
- Kosten
- Änderungen (change control)
\item Erkennung von Engpässen und möglichen Risiken
\item Erarbeitung von Lösungsmöglichkeiten
\item Dokumentation und Berichtwesen
\item Beschaffung von Mitteln zur Realisierung des Projektes (HW, SW, Testzeit...)
\item Projektsteuerung und Zusammenarbeit mit dem Auftraggeber (Vertragstreue)
\end{itemize}

\subsubsection{Forderungen an einen Projektleiter}
\begin{figure}[H]
	\centering

	\includegraphics[width=\textwidth]{forderungen}

\end{figure}	

\textbf{Persönliche Qualifikation:} 
\begin{itemize}
\itemsep0em
	\item Teamgeist
	\item Initiative
	\item Kreativität
	\item Kontaktfähigkeit
	\item Durchsetzungsvermögen
\end{itemize}
\noindent
\textbf{Projektqualifikation:} beinhaltet Kenntnisse und Erfahrungen,
die sich auf die Organisationsmethoden
und -techniken beziehen.
\\
\\
\textbf{Systemqualifikation:} Sie beinhaltet alle Kenntnisse und
Funktion des Projektes. (z.B. Maschine,
Projektleiter sprechen können.
Chemieanlage, Kraftwerk, EDV-Systeme).
\\
\\
\textbf{Führungsstil:} der Projektleiter hat einen kooperativen Führungsstil:
\begin{itemize}
\item Der Mensch steht im Mittelpunk
\item Prinzip der offenen Tür.
\item Führung durch Überzeugung und Argumentation
\item ...
\end{itemize}
\subsection{Ziele}

\textbf{Organisatorische Ziele:}
\begin{itemize}
\item Entwicklung einer flexiblen projektorientierten Organisationsstruktur
mit eindeutiger Regelung der Kompetenzen, Informations- und Entscheidungswege.
\item Klarheit über den Projektablauf bei allen Beteiligten.
\item Sicherstellung der fachübergreifenden Kooperation.
\end{itemize}
\noindent
\textbf{Planungs-Ziele:}
\begin{itemize}
\item Beherrschung des Umfangs und der Komplexität durch inhaltliche Strukturierung.
\item Erzielung einer angemessenen Lastverteilung im Team.
\item Einhaltung von Zeit-, Kosten- und Kapazitätsplanungen.
\end{itemize}
\noindent
\textbf{Controlling-Ziele:}
\begin{itemize}

\item Frühzeitige Erkennung von Planabweichungen und deren zukünftigen Auswirkungen.
\item Aufzeigen von Handlungsalternativen bei Planabweichungen.

\item Dokumentation des Projektverlaufs zur Verwertung in zukünftigen Projekten.
\item Steigerung des Termin- und Kostenbewußtseins bei allen Projektbeteiligten.

\end{itemize}
\noindent
\textbf{Führungsziele:}
\begin{itemize}
\item Steigerung der Produktivität durch:
\begin{itemize}
\item Motivation des Projektteams
\item Konsensbildung durch Entscheidungsdokumentation
\item Erkennen und Lösen von Konfliktsituationen
\item Hohe Akzeptanz durch Anwender-Partizipation
\end{itemize}
\end{itemize}

\subsubsection{Formulierung von Zielen}
Zielen sollten:
\begin{itemize}
	\item SMART:
	\begin{itemize}
		\item \textbf{Spezifisch:} klar und eindeutig
		\item \textbf{Messbar:} wir können das auf eine Schale angeven (Begin mit endzustand vergeleichen)
		\item \textbf{angemessen:} redelijk, realistisch
		\item \textbf{Relevant:}  begrenzt in zeit, man weißt wenn es endet.
	\end{itemize}
	\item lösungsneutral formuliert sein. (do not say how it should be solved)
\end{itemize}

\subsubsection{Zielbeziehungen}

Es gibt drei verschiedene Zielbeziehungen:
\begin{itemize}
\item \textbf{Iterdependenzrelation:} Beeinflussung verschiedener Ziele untereinander: komplementär, konfliktär.
\item \textbf{Präferenzrelation:} Prioritäten von
konkurrierenden Zielen
\item \textbf{Instrumentalrelation:} Ziel-Mittel-Verhältnis von Zielen (the subtarget serves the upper target)
\end{itemize}

\section{Projektverantwortung}

\subsection{Projektsysteme in der Projektorganisation}
\textbf{Politisches Projektsystem:} = politisches Teilsystem des Projektmanagements, also Auftraggeber,
Steuerungsausschuß oder -ausschüsse.
\\
\\
\textbf{Administratives Projektsystem:} = administratives Teilsystem des Projektmanagements, also Projektleitung.
\\
\\
\textbf{Operatives Projektsystem:} = Projektdurchführung (Projektrealisierung).

\subsection{Kooperation}

\begin{figure}[H]
	\centering

	\includegraphics[width=0.66\textwidth]{ch4/projektgruppe}

\end{figure}	

\begin{figure}[H]
	\centering

	\includegraphics[width=0.66\textwidth]{ch4/teilprojekt}

\end{figure}	

\begin{figure}[H]
	\centering

	\includegraphics[width=0.66\textwidth]{ch4/kooperation}

\end{figure}	

\section{Das projektteam und seine Afugaben}

\begin{de}{Projektteam}
Ein Projektteam ist eine interdisziplinäre und hierarchieübergreifende Areitsgruppe, die in der Lage ist, eine bestimmte Aufgabe in Projektform zu lösen.\\
\\
Jedes Mitglied erkennt, dass die Aufgabe nur in einer produktiven
Zusammenarbeit im Team erfolgreich gelöst werden kann und dass dadurch
jedes Teammitglied für seinen Aufgabenbereich selbst profitiert.
\end{de}
\textbf{Warum Teamarbeit?} Einwirkung verschiedener Fachdisziplinen auf komplexe Aufgabenstellungen

\begin{de}{Effizient arbeiten}
die Dinge richtig tun
(in Bezug auf Qualität, Kosten. Termine etc.)
\end{de}

\begin{de}{Effektiv arbeiten}
die richtigen Dinge tun
(in Bezug auf die Arbeitsschritte im Projektablauf,
Prioritäten gemäß Pflichtenheft vs. Änderungen
seitens Auftraggeber etc.)
\end{de}

\begin{de}{Kommunikation im Projektteam}
$$
N = \frac{n(n-1)}{2}
$$
\end{de}

\begin{de}{Soft skills}
weiche, das soziale Umfeld berücksichtigende Fähigkeiten:\\
\\
Tadellose Manieren, vorbildlicher Charakter, Planung von Einstieg und
Aufgabe (Ziel) sowie Beziehungs- und Rückkopplungsnetzwerk (Weg).
\end{de}

\begin{de}{Hard skills}
in Schule und Hochschule erworbene fachliche Fähigkeiten:\\
\\
Techn. und kaufm. Fachkönnen, Wissen in Theorie und Anwendung zur
Erarbeitung exzellenter Lösungen zunehmend schwierigerer Aufgaben.
\end{de}
\noindent
Hard und Soft Skills worden zusammen angewendet auf 3 verschiedene Ebene:
\begin{itemize}
\item \textbf{Fachwissen:} Ein Mitarbeiter muss sein Fachkönnen und -wissen so
anwenden, daß die ihm aufgetragenen ``Dinge richtig'' erledigt werden.
\item \textbf{Organisationswissen:} Ein Vorgesetzter muß dafür Sorge tragen, daß die
``richtigen Dinge'' getan werden.
\item \textbf{Herrschaftswissen:} Ein Mitglied von Vorstand/Geschäftsführung muß
dafür sorgen, daß
\begin{itemize}
	\item im Unternehmen das notwendige Wissen, Können und Motivation
vorhanden sind, um die ``Dinge richtig'' auszuführen.
	\item bei allen Vorhaben (Projekten) klar definiert wird, was die ``richtigen
Dinge'' sind, die zum Wohle der Firma getan werden müssen.
\end{itemize}
\end{itemize}
\subsection{Art des Projektteams}
Projektteams können im Hinblick auf die Zahl und die
Person der Teammitglieder fix oder veränderbar sein.

\begin{itemize}
\item \textbf{Geschlossenes Projektteam:} bei dieser Art eines Projektteams ist geplant, dass vom Anfang bis zum
Ende die Mitarbeiter des Projektes gleich sind, sowohl hinsichtlich ihrer
Zahl als auch ihrer Person.\\
\textbf{Vorteile:} Team Building\\
\textbf{Nachteile:} Arbeitsauslastung, lösungsblind

\item \textbf{Offenes Team:} die Zusammensetzung dieses Projektteams ändert sich im Laufe der
Projektabwicklung. Es werden für die einzelnen Phasen in der Projektarbeit
jeweils unterschiedliche Mitarbeiter und diese in unterschiedlicher Zahl
zur Projektarbeit eingesetzt.\\
\textbf{Vorteile:} Spezialisten\\
\textbf{Nachteile:} Identifikation

\item \textbf{Externes Projektmanagement:} Der Auftraggeber lässt das Projekt ausschließlich durch betriebsfremde
Mitarbeiter planen, kontrollieren und ggf. auch realisieren. D.h. das Projekt-
team besteht nur aus betriebsexternen Mitgliedern.\\
\textbf{Vorteile:} Best Practice (neue frische Ideeen)\\
\textbf{Nachteile:} Fehlende Akzeptanz

\item \textbf{Internes Projektmanagement:} An dem Projekt sind ausschließlich eigene Mitarbeiter des Auftraggebers
beteiligt. D.h. das Team besteht nur aus betriebsinternen Mitgliedern.\\
\textbf{Vorteile:} Prozesskenntnis\\
\textbf{Nachteile:} Veränderungsscheu

\item \textbf{Gemischtes Projektmanagement:} Die Planung und Kontrolle eines Projektes liegt in den Händen eines Teams,
das aus externen und internen Mitgliedern gebildet wird.\\
\textbf{Vorteile:} Kreativität (ook voordelen van intern en extern)\\
\textbf{Nachteile:} Kommunikation
\end{itemize}

\begin{figure}[H]
	\centering

	\includegraphics[width=0.66\textwidth]{ch5/kriterien}

\end{figure}	



\subsection{Aufgaben des Projektteams}
\textbf{Das Projektteam (bzw. die Projektleitung)}

\begin{itemize}
\item \textbf{plant} das ihm anvertraute Projekt im Detail und überwacht dessen Verlauf;
\item \textbf{koordiniert} die Zusammenarbeit aller beteiligten Stellen;
\item \textbf{informiert} die Unternehmens- bzw. Institutsleitung laufend über den Stand
des Projektes, insbesondere bei Schwierigkeiten;
\item \textbf{dokumentier} den Verlauf des Projektes in sachlicher, terminlicher und
finanzieller Hinsicht;
\item \textbf{führt} und \textbf{berät} alle am Projekt beteiligten Mitarbeiter
\item \textbf{bereitet} Entscheidungsunterlagen vor und führt Entscheidungen herbei.
\end{itemize}

\subsection{Kriterien für erfolgreiche Zusammenarbeit im Team}
\begin{enumerate}
\item Kommunikationsfähigkeit
\item Initiative, Engagement, Begeisterungsfähigkeit
\item Integrationsfähigkeit
\item Kontaktfähigkeit
\item Sensibilität, Selbstkontrolle, Verantwortungsbewusstsein, persönliche Integrität
\item Konfliktbewältigunsfähigkeit, Streitkultur
\item Lösungsfähigkeit, ganwheitliches Denken
\item Loyalität, Solidarität, Hilfsbereitschaft
\end{enumerate}

\subsection{Probleme mit der Kommunikation im Team}
\begin{itemize}
\item Blinde Gruppenmoral
\item Rationalisierung
\item Stereotypisierung
\item Selbstzensur
\item Gruppenzensur
\item Meinungswächter
\end{itemize}

\subsection{Stakeholder eines Projektes}
\begin{de}{Projektbeteiligten}

Stakeholder eines Projektes sind alle (natürlichen oder juristischen)
Personen, die ein Interesse am Projekt haben oder vom Projekt in
irgendeiner Weise betroffen sind.

\end{de}

\begin{figure}[H]
\centering
\includegraphics[scale=0.35]{ch5/stakeholder}
\end{figure}

\section{Projektabwicklung, Lastenheft \& Pflichtenheft}

Die Abwicklung von Projekten läuft folgenden Schritten entsprechend ab,
die teilweise zur Strukturgrundlage einer Phasenorganisation werden können:

\begin{enumerate}
\item Beschreibung und Analyse des Problems
\item Beschreibung der zu erreichenden Ziele (\textbf{Zielsetzung})
\item Beschreibung der technischen Lösung (\textbf{Lastenheft)}
\item Beschreibung der fachlichen Lösung (\textbf{Pflichtenheft})
\item umfassende Planung/Entscheidung des Projektablaufs
\item Organisation, Realisierung, Kontrolle und Steuerung des Projektes:
	\begin{itemize}
	\item Erstellung bzw. Erarbeitung der Komponenten für das Gesamtsystem
	\item Zusammenfügung der Systemkomponenten zum Gesamtsystem
	\item Prüfung des Gesamtsystems
	\end{itemize}
\item Einführung des Systems in die Zielumgebung
\item Anwendung des Systems
\end{enumerate}

\subsection{Methodik der Projektabwicklung}
\begin{figure}[H]
	\centering

	\includegraphics[width=0.66\textwidth]{ch6/methodik}

\end{figure}

\begin{enumerate}
\item Wass will ich?
\item Wie soll das erreicht werden?
\item Unter welchen Voraussetzungen, Bedingungen, etc.?
\item Welche Maßnahmen müssen getroffen werde?
\item Welche Hilfsmittel werden benötigt?
\end{enumerate}	
\noindent
(Notiz: schau noch mal die Beispiele in die Folien.)

\section{Projektphasen und Meilensteine}

\begin{de}{Projektphasen}
unterteilen den Projektlebenszyklus in zeitlich
zusammenhängende Abschnitte. Abhängig von den Anforderungen der
jeweiligen Projektart, Branche oder Organisation können diese Aktivitäten
verschieden unterteilt werden.\\
Im Projektphasenplan sollten
\begin{itemize}
\item PhasenZiele
\item produkt- und projektbezogene Phasenergebnisse
\item Phasenabschlüsse
\end{itemize}
festgelegt sein.
\end{de}
Phaseneinteilung bedeuten:
\begin{itemize}
\item \textbf{Schrittweise:} in kleine stukken opdelen, risico minimieren, bessere verständnis
\item \textbf{systemorientierte:} wo sind die probleme
\item \textbf{schematische:} (schemen nutzen) um zeit zu sparen 
\end{itemize}
Vorgehensweise.
\begin{figure}[H]
	\centering
	\caption{Beispiele für ProjektphasenPläne}
	\includegraphics[scale=0.25]{ch7/projektphasenplane}
\end{figure}
\subsection{Projektphasen}
\begin{enumerate}
\item Problemenanalyse
\item Konzeptfindung
\item Definition
\item Entwicklung
\item Realisierung
\item Nutzung
\item Außerdienststellung
\end{enumerate}
\subsection{Hinweise zur Projektdurchführung}
\begin{enumerate}
\item Rechtzeitige Weitergabe von Informationen an Projektbeteiligte
\item Nachweisführung (Projektdokumentation)
\item Festlegung und Überwachung von Verantwortlichkeiten für Arbeitsaspekte
\item Eindeutige Kennzeichnung von Aufgaben und Teilleistungen bzw.
Aufträgen sowie deren sachgerechte Vergabe
\item Überwachung der zu erbringenden Leistungen
\item Prüfung von Entscheidungskompetenzen und ggf. Änderungen erwirken
\item Festlegung sowie Durchführung interner und ggf. externer Projektüber-
wachung/Reviews
\item Projektänderungen und deren Überwachung
\item Beachtung von Schnittstellen
\item Festlegung von Ansprechpartnern und Maßnahmen
zur Inbetriebnahme des Projektes
\end{enumerate}


\section{Projektplanung und regelkreis}

\begin{figure}[H]
	\centering

	\includegraphics[width=0.4\textwidth]{ch8/managementkreis}
	\caption*{Managementkreis für wichtige Funktionen des Projektnanagements}
\end{figure}

\begin{de}{Projektplanung}
ist die vorausschauende Festlegung der Projektdurchführung, es ist keine einmalige Aufgabe zu Beginn sondern ein permanenter Prozess wäherend der Projektabwicklung.\\
Voraussetzungen für die Projektplanung:
\begin{itemize}
\itemsep0em
\item Problemanalyse
\item Ziele des Projektes
\item Konzept
\end{itemize}

Er hat die volgende komponenten:
\begin{itemize}
\itemsep0em
\item Spezifierung der Projektziele
\item Projektstrukturplan
\item Projektorganisation
\item Netzplan und Meilensteinplan (Termine)
\item Kapapzitätsplan
\item Kostenplan
\end{itemize}
\end{de}
\noindent
\textbf{Aufgaben und Ziele der Projektplanung:}
\begin{itemize}

\item höhere Erfolgswahrscheinlichkeit der Zielerreichung
\item die Verminderung des Projektrisikos
\item die Ermittlung der kritischen Arbeitspakete im Projekt
\item den zielgerichteten Einsatz aller Projektressourcen
\item sichere Aussagen zum Projektablauf
\item die reibungslose Koordination aller am Projekt Beteiligten
\item die rechtzeitige Beschaffung notwendiger Ressourcen
\item die laufende Standortbestimmung des Projektes
\item die effiziente Steuerung des Projektes
\end{itemize}

\begin{figure}[H]
	\centering

	\includegraphics[width=0.4\textwidth]{ch8/projektplanung}
	\caption*{Systematische Vorgehensweise bei der Projektplanung}
\end{figure}
\begin{figure}[H]
	\centering

	\includegraphics[width=0.4\textwidth]{ch8/managementkreis}
	\caption*{Planung und Überwachung beim Projektmanagement}
\end{figure}

\subsection{Vorteile und Nachteile von Projektplänen}
\subsubsection{Vorteile}
\begin{itemize}
\item Vorgehen im Projekt wird geordnet und systematisiert
\item Phasenabschlüsse ermöglichen Überprüfung des Projektstatus
\item Phasenmodell erleichtert Projektmitarbeitern die Einarbeitung in das
Projekt
\end{itemize}
\subsubsection{Nachteile}
\begin{itemize}
\item Projektrealität wird in den Phasenmodellen idealisiert
\end{itemize}

\subsection{Projektstrukturplanung}

\begin{de}{Projektstrukturplanung}
verfolgt eine vollständige, aufgabenmäßige,
hierarchische Strukturierung des Projektes. Die Elemente der untersten
Ebenen dieser Zerlegung werden als Arbeitspakete bezeichnet.
\end{de}

\begin{de}{Arbeitspaket}
umfasst alle Tätigkeiten eines Projekts, die sachlich
zusammengehören und in einer organisatorischen Einheit durchgeführt
werden können.
\end{de}

\subsubsection{Projektstrukturplan-Typen}
\begin{itemize}
	\item \textbf{Objektorientierter:} Strukturierung
erfolgt nach den
einzelnen
Bestandteilen
(Objekten /
Produktteile), die im
Rahmen des
Projektes zu
erstellen sind.
	\item \textbf{Funktionsorientierter:} Das Projekt wird in
einzelne
Verrichtungen
(Funktionen) zerlegt,
die für die
Verwirklichung des
Projektes notwendig
sind.
	\item \textbf{Ablauforientierter:} Das Projekt wird
gemäß der logischen
Abfolge seiner
Bestandteile
geordnet.

	\item \textbf{Gemischt-orientierter:} Die vorher
genannten
Verfahren werden
nach praktischer
Zweckmäßigkeit
kombiniert.
\end{itemize}

\begin{figure}[H]
	\includegraphics[width=0.8\textwidth]{strukturplan}
\end{figure}


\subsection{Kostenplanung}
Kostenarten:
\begin{itemize}
\item Personalkosten
\item Materialkosten
\item Transportkosten
\item Sonstige Kosten
\end{itemize}
Ermittlung der Projektkosten: Top-down, Bottom-up, ``Gegenstrom''\\
\\
Eine besondere Herausforderung stellt die Zuordnung von Gemeinkosten
bei der Projektbudgetierung dar.

\subsection{Terminplanung}
Überführung mit Hilfe der anderen Planungsarten des Aufwandes in:
\begin{itemize}
\item Dauern
\item Planung von Pufferzeiten
\item Abstimmung von Meilensteinen
\item Terminierung
\item Optimierung des Projekts
\end{itemize}

\begin{figure}[H]
	\includegraphics[width=0.8\textwidth]{terminplan}
\end{figure}
\subsection{Ressurcenplanung}
\begin{itemize}
\item Bedarf ermitteln
\item Bestimmen der Verfügbarkeiten
\item Vergleich von Bedarf und Verfügbarkeit
\item Kapazitätsausgleich / Optimierung
\end{itemize}

\subsubsection{Ressourcenoptimierung}
Maßnahmen für einzelne Arbeitspakete:
\begin{itemize}
\item Verschiebung
\item Strecken
\item Stauchen
\item Teilen
\end{itemize}

\section{Werkzeuge der Projketplanng und Grundlagen der Netzplantechnik}
\subsection{Das Teufelsquadrat}

\begin{figure}[H]
	\includegraphics[width=0.5\textwidth]{ch9/teufelsquadrat}
\end{figure}

\subsection{Netzplantechnik}
\begin{de}{Netzplantechnik}
sind ein rechnergestütztes aber auch manuell handhabbares Verfahren zur Analyse,
Planung, Kontrolle und Steuerung von komplexen Arbeitsabläufen in F\&E, in der
Fertigung, im Anlagenbau und bei einem oder mehreren zur gleichen Zeit
abzuwickelnden sonstigen Projekten.
\end{de}

\subsubsection{Vorteile der Netzplantechnik}

\begin{enumerate}
	\item Sie liefert einen ausgezeichneten Überblick über die Gesamtheit der Teilvorgänge
eines Projektes und zeigt deren gegenseitigen Abhängigkeiten.
	\item Sie hält dazu an, das gesamte Projekt genau zu durchdenken und frühzeitig
Entscheidungen zu treffen.
	\item Sie ermöglicht eine relativ exakte Vorhersage wichtiger Zwischentermine und des
Endtermins.
	\item Sie weist auf zeitliche Engpässe (kritischer Weg) und Spielräume (Pufferzeiten) hin
und erleichtert es damit, durch gezielte Maßnahmen die geplante Projektdauer
einzuhalten oder zu verkürzen.
	\item Sie schafft die Möglichkeit, auf Störungen, insbesondere Verzögerungen im
Projektablauf, rasch und wirksam zu reagieren.
	\item Sie erleichtert den Vergleich alternativ geplanter Varianten einer Ablaufstruktur.
	\item Sie führt in Verbindung mit der elektronischen Datenverarbeitung zu einer Entlastung
von Routinearbeiten, was sich vor allem bei häufigen Planänderungen auswirkt.

\end{enumerate}

\subsubsection{Probleme bei der Netzplantechnik}

\begin{enumerate}
	\item Der Netzplan ist zu detailliert, was in einem hohen Kontrollaufwand resultiert (zu viele
Aktivitäten).
	\item Der Netzplan wird zu abstrakt aufgebaut und deshalb von den Anwendern (Technikern,
Kaufleute, usw.) nicht verstanden.
	\item Die Netzplanaktivitäten sind nicht kontrollfähig, was zu unrealistischen Plänen führt.
	\item Die EDV-Ergebnisse versteht nur der Planer, woraus ein Vertrauensverlust in die Planung
resultiert.
	\item Es treten starke Verzögerungen bei der Bereitstellung der aktuellen EDV-Ergebnisse auf
(EDV-Zugriffszeit), was zu völligem Unverständnis bei der Projektleitung führt.
	\item 
\end{enumerate}

\section{Mathematische methoden der Projektschätzung}

\begin{minipage}[t]{0.5\textwidth}

\begin{enumerate}
	\item \textbf{Gleichpunktschätzung:} 
		\begin{itemize}
			\item $t_E = \frac{t_o + t_p}{2}$ 
			\item $t_S = \frac{t_p - t_o}{6}$ 
		\end{itemize}
	\item \textbf{Dreipunktschätzung:}
		\begin{itemize}
			\item $t_E = \frac{t_o + 4t_w + t_p}{2}$ 
			\item $t_S = \frac{t_p - t_o}{6}$ 
		\end{itemize}
	\item \textbf{Linearkombination}
\end{enumerate}
\end{minipage}
~
\begin{minipage}[t]{0.49\textwidth}
\begin{enumerate}
	\item \textbf{Gleichverteilung:} 
		\begin{itemize}
			\item $t_E = \frac{t_o + t_p}{2}$ 
			\item $t_S = \frac{t_p - t_o}{6} * \sqrt{3}$ 
		\end{itemize}
	\item \textbf{Dreiecksverteilung:}
		\begin{itemize}
			\item $t_E = \frac{t_o + t_w + t_p}{3}$ 
			\item $t_S = \frac{t_p - t_o}{6}\sqrt{1 + \left(\frac{t_p-t_w}{t_p-t_o}\right)^2 + \left(\frac{t_w-t_o}{t_p-t_o}\right)^2}$ 
		\end{itemize}
	\item \textbf{Betaverteilung}
\end{enumerate}
\end{minipage}

\subsection{Schätzmethoden}
\begin{minipage}[t]{0.5\textwidth}

\subsubsection{Expertenschätzen}
\begin{itemize}
	\item Experten schätzen subjectiv auf Baiss ihrer Erfahrung
	\item Einzel-unde Mehrfachbefragung: es wird auf das Fachwissen einer oder mehrerer Person zurückegriffen
	\item \textbf{Delphi-Methode:} Mehrfachbefragung einer Gruppe von Experten
	\item Schätzklausur: erweitert die Delphi-Methode durch einen offnene Diskussion
\end{itemize}

Let op, deze methodes duren soms te lang en einzelpersonen kunnen fouten maken.

\subsubsection{Vergleichsmethoden}

\begin{itemize}
	\item Basiert auf ähnlichkeiten zwischen abgeschlossenen Projekten und dem aktellen
	\item \textit{Analogiemehtode:} subjektiver, erfahrungsbasierter Vergleich mit bereits abgeschlossenen Projekten
	\item \textit{Relationsmethode:} Erweiterung der Analogiemethods $>$ formalisierter vergleich mit bereits abeschlossenen Projekten.
\end{itemize}

\end{minipage}
~
\begin{minipage}[t]{0.49\textwidth}

\subsubsection{Kennzahlenmethoden}

\begin{itemize}
\item Kennzahlen aus abgeschlossenen Projekten herleiten
\item \textbf{Multiplikatormethode:} Linearer Zusammenhang zwischen Kennzahl und zu schätzender Größe
\item \textbf{Prozentsatzmethode:} Hochrechnung von Schätzungen von Projektteilen auf das gesamte Projekt
\end{itemize}

\subsubsection{Parametrische Methoden / Algorithmisches Schätzverfahren}

\begin{itemize}
	\item Schätzgröße in mathematische Abhängigkeit zu Einflussgrößen
bringen (Korrelation)
	\item \textbf{COCOMO-Methode:} Stellt Programmgröße (Lines of Code) in Beziehung zu Personalaufwand
	\item \textbf{Function Point Analysis:} Verwendet statt Lines of Code Function Points für die Berechnung
\end{itemize}
\end{minipage}

\section{Projektcontrolling}

\begin{figure}[H]
	\centering
	\begin{subfigure}{0.49\textwidth}
		\caption*{Prinzip der Projektsteuerung:}
		\includegraphics[width=\textwidth]{ch11/projektsteuerung}
	\end{subfigure}
	~
	\begin{subfigure}{0.49\textwidth}
		\caption*{Einfaches Modell für das Projektcontrolling:}
		\includegraphics[width=\textwidth]{ch11/controlling}
	\end{subfigure}
		\par\bigskip
		\par\bigskip
		\centering
	\begin{subfigure}{0.45\textwidth}
		\caption*{Vereinfachter Regelkreis beim Projektmanagement (mit Vorkopplung):}
		\includegraphics[width=\textwidth]{ch11/regelkreispm}
	\end{subfigure}
	~
	\begin{subfigure}{0.45\textwidth}
		\caption*{Regelkreis Projektplanung, -überwachung und -steurung}
		\includegraphics[width=\textwidth]{ch11/regelkreis}
	\end{subfigure}
\end{figure}

\subsection{Hauptelemente des Überwachungsregelkreises}

\begin{figure}[H]
	\centering
	\includegraphics[width=0.5\textwidth]{ch11/hauptelemente}
\end{figure}

\begin{enumerate}
	\item \textbf{Erstellung der Vertragsbasis (Auftragsbasis):} 
	Die Vertragsbasis, in der die Projektziele anhand der Systemspezifikation, des Pflichtenheftes,
der Haupt- bzw. Ecktermine und der Projektkosten verankert sind, ist das Fundament für jeden
Managementprozess. Projektänderungen werden an der Vertragsbasis gemessen.

	\item \textbf{Planung und Arbeitsorganisation:} Auf der Basis der vertraglich festgelegten Projektziele sind detaillierte Pläne und eine detaillierte
Arbeitsorganisation aufzubauen. In der Regel werden die Planungsunterlagen (Projekt-
beschreibung, Ablaufpläne, Termin- und Kostenpläne) bereits in der Angebotsphase erstellt und
bei Abschluss des Vertrages (Absichtserklärung, Vorvertrag oder Hauptvertrag) auf den
neuesten Stand gebracht und endgültig verabschiedet.

	\item \textbf{Arbeitsfreigabe:} Die Arbeitsfreigabe ist gewissermaßen der offizielle Anfang des Überwachungsprozesses. Um zu
verhindern, dass Arbeiten ohne Autorisation begonnen werden, ist ein wirkungsvolles Freigabe-
verfahren einzuführen.

	\item \textbf{Arbeitsdurchführung (Leistungserbringung):} Die Durchführung von Projektarbeiten sollte in allen Ebenen und von allen Beteiligten auf klaren
und eindeutigen Planungsunterlagen und in Verbindung mit einer offiziellen Arbeitsfreigabe
erfolgen.

	\item \textbf{Leistungsmessung (Statusermittelung):} In regelmäßigen - meist monatlichen - Abständen, erfolgt die Leistungsmessung (Arbeitsfort-
schritt, Mittelverbrauch und Änderungen).

	\item \textbf{Abweichungsanalyse und Korrekturmaßnahmen:} Die gemessene Leistung (Stufe 5) kann nun unter Bezugnahme auf die Planung und Arbeits-
organisation (Stufe2) analysiert werden. Planungsabweichungen sind zu identifizieren, und ihre
Auswirkungen auf das angestrebte Projektziel müssen bewertet werden. Gegebenenfalls sind
entsprechende Korrekturmaßnahmen zu definieren und an das Management weiterzuleiten.

	\item \textbf{Managemententscheidung:} Die aus der Abweichungsanalyse resultierenden und an das Management als Vorschlag weiter-
geleiteten Korrekturmaßnahmen sind vom Management in Planungsänderungen umzusetzen
oder gegebenenfalls auch abzulehnen. Die Managemententscheidung ist als aktives Stellorgan
im Managementregelkreis zu betrachten. In der Praxis veranlasst die Projektleitung oftmals eine
weiterführende Analyse, um die Auswirkungen geplanter Entscheidungen besser beurteilen zu
können.

	\item \textbf{Änderungen (Änderungsüberwachung):} Bei der Einleitung von Änderungen sind zwei Maßnahmen möglich:
	\begin{itemize}
		\item Vertragsänderungen und
		\item projektinterne Änderungen
	\end{itemize}
	\noindent
	Zeigt die Abweichungsanalyse (Stufe 6), dass die festgestellten Abweichungen durch projekt-
externe Einflüsse, z.B. durch vom Auftraggeber kontrollierte Nahtstellen (verspätete Bereitstellung
von Maschinen, Zulieferungen, usw.) verursacht wurden, so wird dies normalerweise zu
vertraglichen Änderungen (Stufe 8a) führen. Bei projektinternen Einflüssen müssen Änderungs-
vorschläge innerhalb der vertraglichen Regelung unterbreitet werden. Durch die vom Manage-
ment eingeleiteten Änderungen (Stufe 8b) ist der Management-Regelkreis geschlossen.

\subsection{Leistungsforschrittskontrolle}

\subsubsection{0/50/100\%-Methode}

\begin{itemize}
	\item Pauschale Erfassung des Leistungsfortschritt in 3 Stufen:
		\begin{enumerate}
			\item 0\% - Arbeitspaket wurde noch nicht begonnen
		\end{enumerate}
	\item Vorteil:
		\begin{itemize}
			\item Schnelligkeit, Aufwand für Einschätzung gering
			\item Eignet sich bei Projekten mit kurzläufigen Arbeitspaketen und relativ niedrigem Projektrisiko
		\end{itemize}
		
	\item Nachteil:
		\begin{itemize}
			\item Keine differenzierte Abbildung des Fortschritts (es wird
angenommen, dass sich diese Ungenauigkeit über die
Gesamtheit der Arbeitspakete ausgleicht)
			\item Differenzierungsgrad kann für weitere Steuerung nicht
ausreichend sein
		\end{itemize}
\end{itemize}
	
\end{enumerate}


\subsubsection{Meilensteinmethode}

\begin{itemize}
	\item Zur Bestimmung der Leistungsfortschrittskontrolle werden Projekt-
Meilensteine definiert und dienen als Grundlage zur
	\item Vorteil: Sehr differenzierte Methode
	\item Nachteil: Je nach Interpretation der Meilensteine kann es zu Ungenauigkeiten
kommen.
	\item Die Methode ist nur dann anzuraten, wenn die Meilensteine
differenziert planbar sind und zwischen den jeweiligen Meilensteinen
ungefähr gleich große Leistungsabschnitte liegen.
\end{itemize}

\subsection{Kostenverlaufskontrolle}

\subsubsection{Earned Value-Analyse}

$$ EV = (Ferigstellungsgrad)*Plankosten$$

\begin{figure}[H]
	\centering
	\begin{subfigure}{0.49\textwidth}
		\includegraphics[width=\textwidth]{ch11/evagraf}
	\end{subfigure}
	~
	\begin{subfigure}{0.49\textwidth}
		\includegraphics[width=\textwidth]{ch11/kosten}
	\end{subfigure}
\end{figure}

\begin{figure}[H]
	\centering
	\includegraphics[width=\textwidth]{ch11/eva}
\end{figure}




\end{document}