\documentclass[11pt,a4paper]{article}
\usepackage[utf8]{inputenc}
\usepackage[english]{babel}
\usepackage{amsmath}
\usepackage{amsfonts}
\usepackage{amssymb}
\usepackage{graphicx}
\usepackage[left=2cm,right=2cm,top=2cm,bottom=2cm]{geometry}
\author{Ward Schodts}

% Voor frames
\usepackage{mdframed}
% Voor kleuren
\usepackage{xcolor}

\definecolor{solarback}{HTML}{d9d9d9}
\definecolor{solarfront}{HTML}{000000}
\mdfdefinestyle{de}{linewidth=0pt,backgroundcolor=solarback,fontcolor=solarfront}
\newenvironment{de}[1]
{\begin{mdframed}[style=de]\begin{mydef}{\textbf{#1:}}\\} 
{\end{mydef}\end{mdframed}}
\newtheorem{mydef}{Definition}
\usepackage{float}

\newcommand{\tabitem}{~~\llap{\textbullet}~~}

\usepackage{array}
\newcolumntype{P}[1]{>{\centering\arraybackslash}p{#1}}

\begin{document}
\section{Was ist Projektmanagement?}
\subsection{Projekt}

\begin{de}{Projekt (DIN)}
Vorhaben, das im wesentlichen durch  Einmaligkeit der Bedingungen in ihrer Gesamtheit gekennzeichnet ist, wie z.B.:
\begin{itemize}
\setlength\itemsep{0em}
\item Zielvorgabe,
\item zeitlich, finanzielle oder andere Begrenzungen,
\item Abgrenzungen gegenüber anderen Vorhaben,
\item projektspezifische Organisation.
\end{itemize}

\end{de}

\begin{de}{Projekt (allgemein)}
Ein Projekt ist eine besondere, umfangreiche und zeitlich begrenzte
Aufgabe von relativer Neuartigkeit mit hohem Schwierigkeitsgrad und
Risiko, die in der Regel enge fachübergreifende Zusammenarbeit aller
Beteiligten fordert.
\end{de}

\begin{table}[H]
\def\arraystretch{1.2}
\centering
\begin{tabular}{|p{0.2\textwidth}|p{0.7\textwidth}|}

\hline
\textbf{Kriterium} & \textbf{Beschreibung} \\

\hline
Zeitliche Befristung & 
\tabitem Festgelegter Anfangs- und Endzeitpunkt
\\
\hline
Neuartigkeit &
\vspace{-\topsep}
\begin{itemize}
\itemsep0em
\setlength{\itemindent}{-1.3em}
\item Keine sich wiederholenden Routinevorgänge
\item Neue Aufgabestellung mit besonderer Herausforderung
\end{itemize}
\\

\hline
Einmaligkeit &
\vspace{-\topsep}
\begin{itemize}
\itemsep0em
\setlength{\itemindent}{-1.3em}
\item Insgesamt einmaliges Vorhaben
\item Einzelne Aktivitäten innerhalb des Vorhabens können sich jedoch wiederholen
\end{itemize}\\

\hline
Größe & 
\tabitem Umfang müss Einrichtung eines Projektes und einer Projektorganisation rechtfertigen\\
\hline
Komplexität und Zusammenarbeit &
\vspace{-\topsep}
\begin{itemize}
\itemsep0em
\setlength{\itemindent}{-1.3em}
\item Verschiedene untereinander abhänge Teilaufgaben
\item Hoher Abstimmungsbedarf
\item Fächerübergreifende Zusammenarbeit
\end{itemize}\\

\hline
\end{tabular}

\caption{Kriterien für Projekte}
\end{table}

\begin{table}[H]
\centering
\def\arraystretch{1.4}
\begin{tabular}{|P{0.4\textwidth}|P{0.4\textwidth}|}
\hline
\textbf{Routine-Aufgaben} & \textbf{Projekte}\\
wiederholende Tätigkeiten & einmalige Tätigkeiten\\
standardisierbare Tätigkeiten & individuelle Tätigkeiten\\
trainierbare Tätigkeiten & "do it right the first time" Anspruch\\
reproduzierbare Ergebnisse & selten reproduzierbare Ergebnisse\\
systematische Fehler leicht erkennbar  & systematische Fehler schwer erkennbar\\
kontinuierbare Verbesserung möglich & sofortige Korrektur notwendig\\
meist abteilungsspezifische Abwicklung & meis interdisziplinäre Abwicklung\\
meist statische Randbedingungen & meist dynamische Randbedingungen\\
\hline
\end{tabular}
\caption{Differenz zwischen Routine und Projekt}
\end{table}

\subsubsection*{Übung}
Schau die Übungsaufgabe (Welche Vorhaben sind Projekt?) noch mal an!
\subsection{Projektmanagement}

\begin{de}{Projektmangement (DIN)}
Gesamtheit von Führungsaufgaben, -organisation, -techniken und
–mittel für die Abwicklung eines Projektes.
\end{de}

\begin{de}{Projektmanagement (allgemein)}
Projektmanagement ist eine Führungskonzeption für direkte
fachübergreifende Koordination von Planung, Entscheidung,
Realisierung, Überwachung und Steuerung bei der Abwicklung
interdisziplinärer Aufgabenstellungen.
\end{de}

\end{document}